\documentclass{book}

\title{Analysis}
\author{yan}
\date{\today}

\usepackage{amsmath}

\begin{document}

\frontmatter
\maketitle
\tableofcontents

\mainmatter
\part{Limit}
\chapter{Nature Number and Base}
\chapter{Series}

In this part, we will discuss about condensation test of series.

Condensation test means: for any $a, b > 0$, if$f$is a monotonic decreasing and non-negative function on$(a,+\infty)$, then the series
\begin{equation}
\sum_{n=a}^{\infty} f(n) and \sum_{n=a}^{\infty} b^nf(b^n)
\end{equation}
will have same convergency.

\part{Multi Variable Analysis}
\chapter{topology}
A topo
\chapter{Complete Metric Space}


\chapter{Measure}
Measure is a special system on a topology space.
To show the internal structure of `volume` of specific space, let's introduce '$\sigma$-algebra'.

\part{Harmonic Analysis}
\chapter{Fourier Analysisi}
First, consider 1-dimension condition:
$\forall p \in [1, \infty),
\forall f \in L^p([a, a + L]),
\forall \epsilon > 0,
\exists T_n(x) = \sum_{k = -n}^n a_k L^{-1/2} exp(\frac{2 \pi i n(x - a)}{L}),
that |f - T_n|_p < \epsilon$, 
which is to say,
$lim_{n \rightarrow \infty}|f - T_n| = 0$.

Intuitively, we can suspect that under certain condition, since in fourier series,
$a_n = (2L)^{-1/2} \int_{-L}^{L} f(y) exp(\frac{-2 \pi i n y}{2 L}) dy.$
So that
$f(x) = (2L)^{-1} \sum_{n = -\infty}^{\infty} exp(\frac{2 \pi inx}{2L}) \int_{-L}^{L} f(y) exp(\frac{-2 \pi inx}{2L})dy$
which may equal to($\xi = \frac{n}{2L}$)
$f(x) = \int$

\end{document}
