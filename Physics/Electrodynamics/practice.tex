\documentclass{article}

\title{Electric Field Practice}
\author{yan}
\date{\today}

\usepackage{amsmath}

\begin{document}
\part{Field strength and Electric Potential}
1.1 Solve the electric field strength and potential of a point along the perpendicular bisector of a rod that has length of $2l$, carrying a uniform linear charge density of $\lambda$.
Solution 1:


We will solve $E(x)$ first, and then use $V(x) = -\int_{\infty}^{x} E(x) \mathrm{d}l$ to solve $V(x)$.      
\begin{equation}
  \mathrm{d} E(\mathrm{d} Q) = \frac{\mathrm{d} Q}{4 \pi \epsilon_0 \cdot r(\matnrm{d}Q)^2} \cdot cos \theta(\mathrm{d} Q)
\end{eqution}
Which we will build up a integral:
\begin{equation}
  \int_{(l)} \mathrm{d} E(x) =
  \int_{-L}^{L}
  \frac{\lambda \mathrm{d} l}{4 \pi \epsilon_0 (x^2 + l^2)} \cdot
  \frac{x}{\sqrt{x^2 + l^2}}  
\end{equation}
And simplifie it to:
\begin{equation}
  E(x) =
  \int_{-L}^{L} \frac{x \lambda \mathrm{d} l}{4 \pi \epsilon_0 (x^2 + l^2)^\frac{3}{2}}
  = \frac{x \lambda}{4 \pi \epsilon_0} \int_{L}^{-L} \frac{\mathrm{d} l}{(x^2 + l^2)^\frac{3}{2}}
\end{equation}
Solution 2:


We will solve $V(x)$ first, and then use $E(x) = \nabla V(x)$, which, in this 1 dimension problem, is $E(x) = \frac{\partial V(x)}{\partial x}$.
\end{document}
