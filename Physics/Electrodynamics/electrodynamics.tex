\documentclass{article}

\title{Electric Field}
\author{yan}
\date{\today}

\usepackage{amsmath}

\begin{document}

\part{Coulomb's Law and Calculus on Vector Space}
the Coulomb's Law is simple. It shows the way electric charges interact with each other:
\begin{equation}
\mathbf{F}=\frac{1}{4\pi\epsilon_0}\frac{qQ}{r^2}\mathbf{\hat{r}}
\end{equation}
or we sometime write it as:
\begin{equation}
\mathbf{F}=k\frac{qQ}{r^2}\mathbf{\hat{r}}
\end{equation}
If $\mathbf{F}$ refers to the force added to charge A by charge B, $\mathbf{\hat{r}}$ will be a unit ($|\mathbf{\hat{r}}| = 1$) vector point to charge A from charge B. However, directions are easy to decide if you take care of the + - stuff.


Our next goal is to discover the property of electric force. First, to make things easier, we will figure out a concept called electric field to skip a f**king test charge:
\begin{equation}
\mathbf{E}=\frac{1}{4\pi\epsilon_0}\frac{Q}{r^2}\mathbf{\hat{r}}
\end{equation}
or we sometime write it as:
\begin{equation}
\mathbf{E}=k\frac{Q}{r^2}\mathbf{\hat{r}}
\end{equation}
From the equtions, we have $\mathbf{F} = \mathbf{E}q$. It's natural to form a concept of 'field' when the force is directly associated with location. While when such force do exactly the same work between 2 points no matter what route it travles through, and it is continous and differentiable on $R^3$, which means it(the force) is the gradient of a certain function. We call such function the potential function, while it's gradient is called the force. But, how to do calculus on vectors, gradients, different routes in certain space, etc. let's introduce the vector analysis as a tool.

\part{Vector Analysis - Gradient}
For simple calculation on vector space, you can have a look at your linear algebra book.


First, Let's deal with gradient. gradient is a special operator(just like +-*/, however, operating on one target), the $\nabla$ means just like $(\mathbf{\hat x} \frac{\partial}{\partial x} + \mathbf{\hat y} \frac{\partial}{\partial y} + \mathbf{\hat z} \frac{\partial}{\partial z})$. When we solve gradient of a function that give out a number, like $\mathbf{E} : \Re^3 \to \Re$ defined as :

\end{document}
